\documentclass{article}

\usepackage{graphicx}
\usepackage{tabularx}
\usepackage{setspace}
\usepackage{amssymb}
\usepackage{amsmath}
\usepackage[b5paper, portrait, top=2.8cm, bottom=2.8cm, left=2.5cm, right=2.5cm]{geometry}

\begin{document}
	{\setstretch{1.0}
		\thispagestyle{empty}
		\begin{flushleft}
			\includegraphics[width=75px]{lambang-its-color-std.png}\\
			{\fontfamily{cmss}\selectfont
				\vspace{6cm}
				GROUP HOMEWORK\\
				INTRODUCTORY FINANCIAL MATHEMATICS CLASS - SM141432\\
				\vspace{0.2cm}
				\textbf{1st HOMEWORK}\vspace{0.4cm}
				
				\begin{minipage}{10cm}
					\flushleft
					NOVITA PUSPITASARI\\NRP 0611 1440000 011\\ \vspace{0.15cm}
					VENANSIUS RYAN TJAHJONO\\NRP 0611 1440000 043\\ \vspace{0.15cm}
					YUSRIL IZZA FRIZNAINI\\NRP 0611 1540000 048\\ \vspace{0.15cm}
					WINDA FIRDIANA\\NRP 0611 1440000 051\\ \vspace{0.15cm}
				\end{minipage}
				\\
				\vspace{1cm}
				Lecturer: \\
				Endah Rokhmati Merdika Putri, S.Si, M.T, Ph.D\\
				\vspace{1cm}
				\textbf{DEPARTMENT OF MATHEMATICS}\\
				Faculty of Mathematics, Computing, and Data Science\\
				Sepuluh Nopember Institute of Technology\\
				Surabaya 2018\\
			}
		\end{flushleft}
	}
	\pagebreak
	\textbf{Questions}\\
	\begin{enumerate}
		\item Taken from General Interest Rate, what is happening if $m\rightarrow\infty$? Clue: derive the equation $\displaystyle\lim_{m\rightarrow\infty}P_0\big(1+\frac{r}{m}\big)^{mn}$.
		\item Suppose that you are to receive payments (in thousands
		of dollars) at the end of each of the next five years. Which of the following
		three payment sequences is preferable? 
		\begin{enumerate}
			\item [\textbf{A.}] 12, 14, 16, 18, 20
			\item [\textbf{B.}] 16, 16, 15, 15, 15
			\item [\textbf{C.}] 20, 16, 14, 12, 10
		\end{enumerate}
		Do your calculation for $r=5\%$, $r=10\%$, and $r=15\%$. Answer it with detail explanation, as a Lender or as a Buyer.
		\item A company needs a certain type of machine for the next
		five years. They presently own such a machine, which is now worth
		\$6,000 but will lose \$2,000 in value in each of the next three years, after
		which it will be worthless and unuseable. The (beginning-of-the-year)
		value of its yearly operating cost is \$9,000, with this amount expected
		to increase by \$2,000 in each subsequent year that it is used. A new machine
		can be purchased at the beginning of any year for a fixed cost of
		\$22,000. The lifetime of a new machine is six years, and its value decreases
		by \$3,000 in each of its first two years of use and then by \$4,000
		in each following year. The operating cost of a new machine is \$6,000 in its first year, with an increase of \$1,000 in each subsequent year. If the
		interest rate is 10\%, when should the company purchase a new machine? 
	\end{enumerate}

	\pagebreak
	\textbf{Solutions}
	\begin{enumerate}
		\item Will be derived the equation $\displaystyle\lim_{m\rightarrow\infty}P_0\left(1+\frac{r}{m}\right)^{mn}$ for knowing the Interest Rate formula for $m\rightarrow\infty$.\vspace{0.5cm}\\
		\begin{tabular}{cll}
			$\displaystyle\lim_{m\rightarrow\infty}P_0\left(1+\frac{r}{m}\right)^{mn}$&=&$P_0\displaystyle\lim_{m\rightarrow\infty}\left(1+\frac{r}{m}\right)^{mn}$\vspace{0.15cm}\\
			&=&$P_0\displaystyle\lim_{m\rightarrow\infty}\left(1+\frac{1}{\frac{m}{r}}\right)^{mn}$\vspace{0.15cm}\\
			&=&$P_0\displaystyle\left(\lim_{m\rightarrow\infty}\left(1+\frac{1}{\frac{m}{r}}\right)^{\frac{m}{r}}\right)^{rn}$\vspace{0.15cm}\\
		\end{tabular}\\
		Then, for any $0<r<1$, we know $\displaystyle p=\frac{m}{r}\rightarrow\infty$, then\vspace{0.15cm}\\
		\begin{tabular}{cll}
			$P_0\displaystyle\left(\lim_{m\rightarrow\infty}\left(1+\frac{1}{\frac{m}{r}}\right)^{\frac{m}{r}}\right)^{rn}$&=&$P_0\displaystyle\left(\lim_{p\rightarrow\infty}\left(1+\frac{1}{p}\right)^{p}\right)^{rn}$\vspace{0.15cm}\\
		\end{tabular}\\
		By using the Euler formula $\displaystyle\lim_{x\rightarrow\infty}\left(1+\frac{1}{x}\right)^x=e$, we can conclude that \\ 
		\begin{equation*}
		\fbox{%
			$\displaystyle\lim_{m\rightarrow\infty}P_0\left(1+\frac{r}{m}\right)^{mn}=P_0e^{rn}$
		}
		\end{equation*}
		
		\item Firstly, we have to find the present value based on the case. For doing calculation, we used Microsoft Excel. Using formula $PV=FV(1+r)^{-n}$ for $r=5\%$, $r=10\%$, and $r=15\%$\vspace{-0.5cm}\\
		\begin{itemize}
			\item for $r=5\%$
			\begin{center}
				\begin{tabular}{|c|c|c|c|c|c|c|}
					\hline
					\textbf{Present Value }&\textbf{1}&\textbf{2}&\textbf{3}&\textbf{4}&\textbf{5}&\textbf{Total PV}\\\hline
					\textbf{A}&11.43&12.70&13.82&14.81&15.67&68.43\\\hline
					\textbf{B}&15.24&14.51&12.96&12.34&11.75&66.80\\\hline
					\textbf{C}&19.05&14.51&12.09&9.87&7.84&63.36\\\hline
				\end{tabular}
			\end{center}
			\item for $r=10\%$
			\begin{center}
				\begin{tabular}{|c|c|c|c|c|c|c|}
					\hline
					\textbf{Present Value }&\textbf{1}&\textbf{2}&\textbf{3}&\textbf{4}&\textbf{5}&\textbf{Total PV}\\\hline
					\textbf{A}&10.91&11.57&12.02&12.29&12.42&59.21\\\hline
					\textbf{B}&14.55&13.22&11.27&10.25&9.31&58.60\\\hline
					\textbf{C}&18.18&13.22&10.52&8.20&6.21&56.33\\\hline
				\end{tabular}
			\end{center}
			\item for $r=15\%$
			\begin{center}
				\begin{tabular}{|c|c|c|c|c|c|c|}
					\hline
					\textbf{Present Value }&\textbf{1}&\textbf{2}&\textbf{3}&\textbf{4}&\textbf{5}&\textbf{Total PV}\\\hline
					\textbf{A}&10.43&10.59&10.52&10.29&9.94&51.78\\\hline
					\textbf{B}&13.91&12.10&9.86&8.58&7.46&51.91\\\hline
					\textbf{C}&17.39&12.10&9.21&6.86&4.97&50.53\\\hline
				\end{tabular}
			\end{center}
		\end{itemize}
	Secondly, based on the result, we can conclude :
	\begin{itemize}
		\item As a Lender, the best option is :
		\begin{itemize}
			\item[a.] For $r=5\%$, take option \textbf{A}
			\item[b.] For $r=10\%$, take option \textbf{A}
			\item[c.] For $r=15\%$, take option \textbf{B}
		\end{itemize}
		\item As a Buyer, the best option is : 
		\begin{itemize}
			\item[a.] For $r=5\%$, take option \textbf{C}
			\item[b.] For $r=10\%$, take option \textbf{C}
			\item[c.] For $r=15\%$, take option \textbf{C}
		\end{itemize}
	\end{itemize}
	\vspace{0.5cm}
	After doing some calculation, our group has found few interested things. Let's take a look for $r=20\%$ and $r=30\%$
	\begin{itemize}
		\item for $r=20\%$
		\begin{center}
			\begin{tabular}{|c|c|c|c|c|c|c|}
				\hline
				\textbf{Present Value }&\textbf{1}&\textbf{2}&\textbf{3}&\textbf{4}&\textbf{5}&\textbf{Total PV}\\\hline
				\textbf{A}&10.00&9.72&9.26&8.68&8.04&45.70\\\hline
				\textbf{B}&13.33&11.11&8.68&7.23&6.03&46.39\\\hline
				\textbf{C}&16.67&11.11&8.10&5.79&4.02&45.69\\\hline
			\end{tabular}
		\end{center}
		\item for $r=30\%$
		\begin{center}
			\begin{tabular}{|c|c|c|c|c|c|c|}
				\hline
				\textbf{Present Value }&\textbf{1}&\textbf{2}&\textbf{3}&\textbf{4}&\textbf{5}&\textbf{Total PV}\\\hline
				\textbf{A}&9.23&8.28&7.28&6.30&5.39&36.49\\\hline
				\textbf{B}&12.31&9.47&6.83&5.25&4.04&37.89\\\hline
				\textbf{C}&15.38&9.47&6.37&4.20&2.69&38.12\\\hline
			\end{tabular}
		\end{center}
	\end{itemize}
		Based on the table above, we can conclude :
		\begin{itemize}
			\item As a Lender, the best option is :
			\begin{itemize}
				\item[a.] For $r=20\%$, take option \textbf{B}
				\item[b.] For $r=30\%$, take option \textbf{C}
			\end{itemize}
			\item As a Buyer, the best option is : 
			\begin{itemize}
				\item[a.] For $r=20\%$, take option \textbf{C}
				\item[b.] For $r=30\%$, take option \textbf{A}
			\end{itemize}
		\end{itemize}\vspace{0.5cm}
		From all of the results we had, our group have hypothesis that \textbf{for larger $r>0$}, the best option for a Lender is option C and the best option for a Buyer is option A.
		
		\pagebreak
		\item Let's take a look for the case. It describes :
		\begin{itemize}
			\item Old Machine
			\begin{itemize}
				\item[a.] Worth for \$6,000 and will lose in value \$2000 yearly, then worthless and unusable.
				\item[b.] Its operating cost is \$9,000 yearly and increases \$2,000 in each year that it is used.
			\end{itemize}
			\item New Machine
			\begin{itemize}
				\item[a.] Can be bought for \$22,000 in the beginning of any year.
				\item[b.] Has 6 years lifetime. Its value decreases by \$3,000 in each of first two years and \$4,000 in the next following year.
				\item[c.] Has operating cost \$6,000 in its first year and increases \$1,000 in each subsequent year.
			\end{itemize}
		\end{itemize}
		From the information above, \textbf{there are only four types of assumptions} because the old machine's life is three years starts from 0, 1, 2, 3. Afterwards, let's take a look for case 3 when the company bought the machine in the beginning of year 3.
		\begin{itemize}
			\item First cash flows\\
			The company doesn't buy a new machine, but use its old machine, so we will have \$9,000 for its operating cost. \textbf{[\$9,000]}
			\item Second cash flows\\
			In year 2, the company still operating its old machine, so we will have $\$9,000+\$2,000$ for its operating cost. \textbf{[\$11,000]}
			\item Third cash flows\\
			In year 3, the company decides to buy a new machine. It means that the old machine still worth for $\$6,000-\$2,000-\$2,000$ (The company sells the machine and get \$2,000). Knowing from the case, a new machine is purchased for \$22,000 and its operating cost is \$6,000. \textbf{[\$26,000]}
			\item Fourth cash flows\\
			In year 4, the company use its new machine which they bought last year, so the company must pay $\$6,000+\$1,000$ for its operating cost. In the other hand, its new machine will lose in value \$3,000 becomes \$19,000. \textbf{[\$7,000]}
			\item Fifth cash flows\\
			In year 5, the company use its new machine which they bought last two year, so the company must pay $\$6,000+\$2,000$ for its operating cost. In the other hand, its new machine will lose again in value \$3,000 becomes \$16,000. \textbf{[\$8,000]}
			\item Sixth cash flows\\
			In year 6, the company decided to sell it's machine and its worth for $\$16,000-\$4,000=\$12,000$. So, the company will received \$12,000. \textbf{[-\$12,000]}
		\end{itemize}
	From the explanation above, we have the sequence \textit{(in thousand dollars)} as written in the textbook \{9, 11, 26, 7, 8, $-$12\}. With the same way, we have the six-year cash flows for case 1, 2, and 4 as follows:\vspace{-0.5cm}\\
	\begin{itemize}
		\item Case 1: \{22, 7, 8, 9, 10, $-$4\}
		\item Case 2: \{9, 24, 7, 8, 9, $-$8\}
		\item Case 3: \{9, 11, 26, 7, 8, $-$12\}
		\item Case 4: \{9, 11, 13, 28, 7, $-$16\}
	\end{itemize}
	To analyze when the company should buy the machine, we use present analysis formula where $PV=FV(1+r)^{-n}$. By applying the formula to all sequences above and we will have the results in table as follows.\vspace{-0.5cm}
	\begin{center}
		\begin{tabular}{|c|c|c|c|c|c|c|c|}
			\hline
			\textbf{Present Value }&\textbf{1}&\textbf{2}&\textbf{3}&\textbf{4}&\textbf{5}&\textbf{6}&\textbf{Total PV}\\\hline
			\textbf{A}&22.00&6.36&6.61&6.76&6.83&$-$2.48&46.083\\\hline
			\textbf{B}&9.00&21.82&5.79&6.01&6.15&$-$4.97&43.794\\\hline
			\textbf{C}&9.00&10.00&21.49&5.26&5.46&$-$7.45&43.760\\\hline
			\textbf{D}&9.00&10.00&10.74&21.04&4.78&$-$9.93&45.627\\\hline
		\end{tabular}
	\end{center}
	From the result, we have decided the company must buy a new machine at the beginning of year 3 \textit{(two years from now)}.
	\end{enumerate}
	\vspace{10cm}
	\textit{\textbf{Note: Excel file for calculation is included in the email.}}
\end{document}